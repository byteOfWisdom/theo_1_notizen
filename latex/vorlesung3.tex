\title{Vorlesung 3}

\section{Im Allgemeinen (1-Dim)}

\eq{1dForce}{m \ddot x = F(x, \dot x, t)}

ist eine gewoehnliche DIfferentialgleichung 2. Ordnung => braucht 2 Randbedingungen zum Loesen.

Am einfchsten: lege $x(t_0), \dot x(t_0)$ fest, dann ist $x(t)$ eindeutig bestimmt.

Numerische Loesung:

Fange an bei $t = t_0$. $a(t_0) = {F(x(t_0), \dot x(t_0), t_0) \over m}$

Zur Zeit $t = t_0 + \Delta t$, mit $\Delta t$ inifinitesimal: 

\eq{}{v(t_1) = v(t_0) + \Delta t \cdot a(t_0)}
\eq{}{x(t_1) = x(t_0) + \Delta t \cdot v(t_0)}
\eq{}{a(t_1) = {F(x(t_1), \dot x(t_1), t1) \over m}}
etc.\\


Laplacesche Behauptung:
Gib mir die Koordinaten \& Geschwindigkeiten aller Koerper in einer feten Zeit und ich kann die Zukunft vorhersagen.\\

Aber:
\begin{enumerate}
	\item man kennt nicht alle kraefte
	\item in vielen Faellen haengt die Bahnkurve sehr sensitiv von den Anfangsbedingungen ab:
	Falls $x(t_0) -> x(t_0)(1 + \epsilon), |\epsilon| << 1, {x'(t) - x(t) \over x(t)}$ kann recht bald $O(1)$ werden. \footnote{$O(1)$ ist die Groessenordnung von 1, bedeutet hier, dass der Fehler 100\% wird.}

	 ``chaotische'' Systeme, ``deterministisches Chaos''

	 Beachte: $X(t_0), \dot x(t_0)$ immer nur mit endlichen Genauigkeiten bekannt.
\end{enumerate}

\subsection{Methoden zum Loesen der Bewegungsgleichung (1-dim)}

\subsubsection*{F haengt nur von x ab}
\eq{}{m \ddot x = m {dv \over dt} = F(x)}
Kettenregel:
\eq{}{
	{dv \over dx} 
	= {dv \over dx} {dx \over dt} 
	= v \cdot {dv \over dx}
}

\eq{}{mv {dv \over dx} = F(x) \text{Trennung der Variablen}}

\eq{}{
	&m \int_{v_0}^{v} v' dv' 
	= \int_{x_0}^{x} F(x') dx'\\
	&=> {1\over 2} m (v^2(x) - v^2(x_0)) \\
	&=> v(x) = \pm [v^2(x_0) + {2\over m} \int_{x_0}^{x} F(x')dx']^{1 \over 2}\\
	&=> \pm \int_{x_0}^{x} {dx' \over [{2\over m} \int_{x_0}^{x} F(x')dx' + v^2(x_0)]^{1 \over 2}]^{1\over 2}}
}

\subsubsection*{F haengt nur von v ab}

\eq{}{
	m {dv \over dt} = F(v)\\
	=> m \int_{v_0}^{v} {dv' \over F(v')} 
	= \int_{t_0}^t dt' = t - t_0
}

loese nach $v(t)$ auf:
\eq{}{
	x(t) = x(t_0) + \int_{t_0}^{t} v(t') dt'
}

\subsubsection*{F haengt nur von t ab}

\eq{}{
	m {dv \over dt}
	= F(t)
}

\eq{}{
	m [v(t) - v(t_0)]
	= \int_{t_0}^{t} F(t') dt'
}

\eq{}{
	x(t) = x(t_0) + v(t_0) \cdot (t - t_0)
	+ {1 \over m} \int_{t_0}^{t} dt' 
	\int_{t_0}^{t} dt'' F(t'')
}

\subsubsection*{``Durch geschickten Ansatz''}

Beispiel:\\
Angetriebener, gedaempfter harmonischer Oszillator.\footnote{hier kommt eine tolle skizze von einem harmonischen oszillator hin}

$x$: Auslenkung aus Ruhelage (fuer $F_{ext} = 0$)

\eq{I34}{
	m \ddot x = - k x - b \dot x + F \cos(\omega t)\\
	=> \ddot x + 2 \gamma \dot x + \omega^2 x
	= f \cos(\omega t)
}

mit
\eq{gamma_def}{
	\gamma = {b \over 2 m}
}

\eq{omega_def}{
	\omega = \sqrt{{k \over m}}
}

\eq{f_def}{
	f = {f \over m}
}


Inhomogene gewoehnliche Differenzialgleichung. Allgemeine Loesung.

\eq{}{
	x(t) &= [\text{allgemeine Loesung d. homogenen DiffGl.}]\\ 
	&+ \text{spezielle Loesung der homogenen Gleichung}
}


Fuer spezielle Loesung definiere Komplexe Beschleunigung $f e^{i \omega t}$ und komlexe Koordinate $z$, mit
\eq{I36}{
	\ddot x + 2 \gamma \dot z + \omega_0^2 z 
	= f e^{i \omega t}
}

Funktioniert, da \ref{eq:I34} linear ist: Keine Terme mit $x^2, \dot x^2$ etc.

Ansatz:
\eq{I37}{
	z{t} = {f \over R} e^{i \omega t}, R = const
}

\eq{helper}{
	\dot z(t) 
	= i \omega z(t), \ddot z(t) 
	= - \omega^2 \cdot z(t)
}

durch einsetzen \ref{eq:helper} in \ref{eq:I36} ergibt sich:
\eq{}{
	[- \omega^2 + 2 i \omega \gamma + \omega_0^2]
	= 1\\
	=> R = \omgea_0^2 + \omega^2 + 2 i \omega \gamma
	= r e^{i \theta}, r, t \in \Re
}

\eq{I39}{
	\tan \theta = {Im(R) \over Re(R)}
	= {2 \omega \gamma \over \omega_0^2 - \omega^2}
}

Beachte: \ref{eq:I37} ist nicht die allgemeinste Loesung. Es wird weiterhin eine Loesung der homogenen Gleichung benoetigt um die Anfangsbedingungen zu erfuellen.

Fuer feste f: maximale Auslenkung ${f \over r}$ am groessten wenn $r^2$ minimiert wird.

\eq{I40}{
	&\diff{r^2}{\omega^2} = 2 (\omega^2 - \omega_0^2)
	+ 4 \gamma^2
	= \sigma\\
	&=> \omega_r^2 = \omega_0^2 - 2 \gamma^2
} 

\eq{I41}{
	r^2(\omega_r) = 4 \gamma^2 (\omega_0^2 - \gamma^2)
}

Fuer schwache Daempfung, $\gamma^2 << \omega_0^2: \omega_r \approx \omega_0:$ Eigenfrequenz der Schwingung:
\eq{}{
	\tan \theta 
	= {2 \omega \gamma \over \omega_0^2 - \omega^2}
	= {2 \omega \gamma \over (\omega_0 + \omega)(\omega_0 - \omega)}
	\approx \footnote{$\omega_r \approx \omega_0$}
	{\gamma \over \omega_0 - \omega}
}

Allg Loesung: fuer $\gamma < \omega_0$: unter-kritisch
\eq{I42}{
	x(t) = C e^{-\gamma t} \cos(\Omega t + \alpha)
	+ {
		f \over [(\omega_0^2 - \omega^2)^2 
		+ 4 \gamma^2 \omega^2]^{1\over 2}
	} \cos[\omega t - \arctan({2 \gamma \omega \over \omega_0^2 - \omega^2})]
}

\eq{}{
	\Omgea = \sqrt{\omega_0^2 - \gamma^2}
}

$C, \alpha$: Integrationskonstanten, (-> Anfangsbedingungen)

Fuer externe Kraft $F_{ext} = m f \sin(\omega t)$ Loesung aus Imaginaer-Teil von \ref{eq:I37}

Beliebige periodische externe Kraft: Durch Fourier-Zerlegung:
\eq{I43}{
	{1 \over m} F_{ext}(t) = \sum_{n=1}^\alpha 
	[f_n \cos(n \omega t) + \~f_n \sin(n \omega t)]
}

Da \ref{eq:I36} linear ist: einsetzen in \ref{eq:I42} die spezielle Loesung durch entsprechende Summe spezieller Loesungen.

\section{Kinetische und potenzielle Energie}

Zunaechst in einer Dimension, fuer Koerper mit konstanter Masse:

\eq{}{
	m \ddot x 
	= m \diff{v}{t} 
	= \ddt (m v)
	= F(x, v, t)
}

multiplizieren mit v

\eq{}{
	&v \ddt (m v) = v F(x, v, t)\\
	&\ddt ({1\over 2} m v^2)
}

Integrieren nach t ergibt

\eq{II1}{
	{1 \over 2} m (v^2(t_2) - v^2(t_1))
	= \int_{t_1}^{t_2} \diff{x}{t} F(x, v, t)
	= \int_{x(t_1)}^{x(t_2)} F(x, v, t) dx
}

Definiere Kinetische Energie:
\eq{II2}{E_{kin} = {1\over 2} m v^2}

Die Gleichung \ref{eq:II1} besagt: Aenderung der kinetischen Energie entspricht der geleisteten Arbeit.

\eq{II3}{
	W = \int_{x_1}^{x_2} F(x, v(x), t(x)) dx
}

Im Allgemeinen muessen wir $x(t)$ kennen um \ref{eq:II3} berechnen zu koennen; kenntnis von $x_1 = x(t_1)$ und $x_2 = x(t_2)$ ist nicht ausreichend. 

Wichtiger Spezialfall: F haengt nur von  ab.

=> definiere potenzielle Energie 
\eq{II4}{
	V(x) = \int_{x_n}^{x} F(x') dx'
}

In diesem Fall:
\eq{II5}{
	\int_{x_1}^{x_2} F(x) dx = V(x_2) - V(x_1)\\
	=> \ref{eq:II1} E_{kin}(t_2) - E_{kin}(t_1)
	= V(x_2) - V(x_1)
}

\eq{II6}{
	=> E_{kin}(t_1) + V(x(t_1)) = E_{kin}(t_2) + V(x(t_2))\\
	=> E_{tot} = E_{kin} + V \text{ist erhalten! Bedingung: F haengt nur von x ab}
}
