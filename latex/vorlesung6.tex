\section{Vorlesung 6}

\subsection{Allgemeiner Fall: Verschieden Massen und k}

\newcommand{\w}{\omega{def}}

\eq{II42}{
	&\ddot x_1 + \w_{1, 1}^2 x_1 + \w_{1, 2}^2 = 0\\
	&\ddot x_2 + \w_{2, 2}^2 x_2 + \w_{1, 2}^2 = 0
}

Spezialfall: 
\eq{}{
	w_{1, 1}^2 &= \w{1, 2}^2 = {k + \kappa \over m} (> 0)\\
	w_{1, 2}^2 &= - {k \over m} (< 0)
}

Suche nach komplexen Loesungen 
\eq{II43}{
	x_1(t) = x_1 e^{\w t}\\
	c_1 \in C
}

Gleichung \ref{eq:II43} in \ref{eq:II42} einsetzen:

\eq{}{
	()
}

In Matrixform:

\eq{II44}{
	\mat{
		\w_{1, 1}^2 - \w^2 & \w_{1, 2}^2 \\
		\w_{1, 1}^2 & \w_{1, 2}^2 - \w^2
	} \cdot 
	\mat{c_1 \\ c_2} 
	= 0
}

Lineare homogen Gleichung \rightarrow [unleserlich]

\eq{II45}{
	[fehlt]
}

Die Loesungen \ref{eq:II45} sind die Eigenwerte der Matrix $\mat{\w_{1, 1}^2 & \w_{1, 2}^2 \\ \w_{1, 1}^2 & \w_{1, 2}}$. Die Gleichung \ref{eq:II44} kann auch als Eigenwert geschrieben werden.


\eq{II46}{
	\mat{
		\w_{1, 1}^2 & \w_{1, 2}^2 \\
		\w
	}
}

muss ich nachher irgendwo abschreiben.



\section{Lagrange und Hamilton Formalismus}

Ziel: Systematische Herleitung der Bewegungsgleichungen in verallgemeinerten Koordinaten.

Vorteile: 
\begin{itemize}
	\item Fuer konservative Systeme: Relativ einfache Herleitung der Bewegungsgleichung auch in nicht Kartesischen Koordinaten.

	\item Grundlage beinahe aller modernen THeoretischen Physik (Quantenmechanik, Quantenfeldtheorie...)
\end{itemize}

Beachte: Die resultierenden Bewegungsgleichungen sind equivalten zu den newtonschen
\rightarrow lineare Superposition

\subsection{Verallgemeinerte Koordinaten}
Fuer N-Koerper um 3D-Raum Brauchen wir 3N Koordinaten.

Kartesisch: \eq{iii1}{\x_i(t) = (x(t), y(t), z(t)); i = 1...N}

Sind meistens nicht so bequem \rightarrow verallgemeinerte Koordinaten


\eq{iii2}{
	\{q_k\} &\text{mit } \x_i = \x_i(q_k, t) \\
	&\text{also } q_k = q(\x_i, t)
}

Beachte: 
\begin{itemize}
	\item $q_k$ brauchte keine Laenge zu sein. oft sind Winkel bequemer

	\item Bezeichnung \ref{eq:iii2} koennen explizit von der Zeit abhaengen
\end{itemize}


Lagrange-Gl: Differenzialgleichungen 2. Ordnung fuer $q_k(t)$

\subsection{Ein Koerper in einer Dimension}

\eq{iii3}{
	q(t) = q [ x(t), t]
}

Nach x aufgeloest:

\eq{iii4}{
	x(t) = x[q(t), t]
	\x(t) 
	= \pdiff{x}{q} \cdot \diff{q}{t} 
	+ \pdiff{x[q, t]}{t}
}
\footnote{$\diff{q}{t}$ ist eine verallgemeinerte geschwindigkeit}

Im Lagrange-Formalisumus werden $q$ und $\dot q$ als unabhaengige Variablen behandelt. Also $\pdiff{q}{\dot q} = 0$ und $\pdiff{\dot q}{q} = 0$

Linearer Impuls:

\eq{iii6}{
	p_x = m \cdot \dot x
	= \diff{}{x} (\half m \dot x^2)
	= \diff{E}{x}
}

Analog verallgemeinerter Impuls:

\eq{iii7}{
	p_q &= \pdiff{E(q, \dot q, t)}{\dot q}
	= \pdiff{}{q} [\half m \dot x(q, \dot q , t)]\\
	&= \diff{E}{\dot x} \pdiff{\dot x}{\dot q}
	= p_x \pdiff{x}{q} \\
	&= \pdiff{x}{q} (???)
}

Gleichung \ref{eq:iii7} nach der Zeit abgeleitet:

\eq{iii8}{
	\dot p_q = \dot p_x \pdiff{x}{q} 
	+ p_x \diff{}{t} (\pdiff{x}{q})
}

\eq{iii9}{
	\diff{}{t} (\pdiff{x}{q}) = (\pdiff{}{q} \pdiff{x}{q} ) \dot q + \pdiff{}{t} (\pdiff{x}{q})
}

\eq{iii10}{
	\pdiff{}{x} \rightarrow
	\pdiff{}{q} \dot x 
	= \pdiff{}{q} \pdiff{x}{q} 
	+ \pdiff{}{q} \pdiff{x}{t}
	= \diff{}{t} (\pdiff{x}{q})
}

Gl \ref{eq:iii10} in \ref{eq:iii8}

\eq{}{
	\dot p_q 
	= \dot p_x \pdiff{x}{q} 
	+ p_x \pdiff{}{q} \dot x
	= F_x \pdiff{x}{q} + p_x \pdiff{\dot x}{q}
}

\eq{iii11}{
	= F_x \pdiff{x}{q} + \diff{E_kin}{\dot x} \pdiff{x}{q} = Q + \pdiff{E_kin(\dot x(q, \dot q, t))}{q}
}


\eq{iii12}{
	\text{mit } Q(q, \dot q, t) = F_x(x, \dot x, t) \pdiff{x}{q} \text{ : Verallgemeinerte Kraft}
}

Koennen Kraft F aufspalten in konservativen Teil und Rest:

\eq{iii13}{
	F_x(x, \dot x, t) = - \diff{V(x)}{x} + \tilde F(x, \dot x, t)\\
	Q(q, \dot q, t) = - \diff{V}{x} \pdiff{x}{q}
	+ \tilde F_x(x(q, t), \dot x(q, \dot q, t), t)
}

\eq{iii14}{
	= - \pdiff{V(q, t)}{q} + \tilde Q
}

Beachte: V als Funktion von q, t hat (natuerlich) andere funktionale Form als V(x).

z.B.: \eq{iii15}{
	V(t) = \half k q ????????
}

Gleichung \ref{eq:iii14} in \ref{eq:iii11}:

\ref{iii16}{
	\dot p_q = \pdiff{}{q} (E_{kin} - V)+ \tilde Q
	= \pdiff{L}{q} + \tilde Q
}

Lagrange Funktion
\eq{iii17}{
	L = E_{kin} (q, \dot q, t) - V(q, t)
}

Gleiche Form durch mehrere Koerper.


Fuer $$, $$

\eq{iii18}{
	q &= x^2, x = \sqrt{q} \rightarrow \pdiff{x}{q} = {1 \over 2 \sqrt{q}}\\
	&\rightarrow {1 \over 2 \sqrt{x}} \dot q 
	\rightarrow E_{kin} = {m \over 2} \dot q^2 {1 \over 4 q} = {m \dot q^2 \over 8 q}
}

\eq{}{
	p_q = m \dot x {1 \over 2 \sqrt{q}} = {m \dot q \over 4 q}
}

\eq{}{
	\ddt ({m \dot q \over 4 q}) 
	= \pdiff{}{q} ({m \dot q^2 \over 8 q} - \half k q)
	= - {m \dot q^2 \over 8 q^2} - \half k\\
	{m \ddot q \over 4 q} = {m \dot q^2 \over 4 q^2}
}

[hier fehlt die letzte gleichung]